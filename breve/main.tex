\documentclass{book}
\usepackage{import}
\import{/home/vatican/communionem-infirmorum/packages}{packages}
\geometry{a5paper, hdivide={1cm,*,1cm}, vdivide={1cm,*,1cm}}
\begin{document}
\pagestyle{empty}
\begin{center}
    \large Rito Breve da Comunhão dos Enfermos
\end{center}
\begin{flushleft}
    \textcolor{red}{Usa-se este rito mais breve quando a Sagrada Comunhão é dada a muitos doentes em vários quartos sa mesma casa, se for conveniente, alguns elementos do rito ordinário.}
    \vspace{.1cm} \\
    \textcolor{red}{O Rito pode principiar na igreja, na sacristia ou no quarto do primeiro doente, dizendo o ministro a seguinte antífona:}
    \vspace{.1cm} \\
    Ó sagrado banquete de que somos os convivas, \\
    no qual recebemos o Cristo em comunhão! \\
    Nele se recorda a sua paixão, \\
    nosso coração se enche de graça \\
    e nos é dado o penhor da glória que há de vir.
    \vspace{.1cm} \\
    \textcolor{red}{Outras antífonas à escolha:}
    \vspace{.1cm} \\
    Quão suave, Senhor, é a ternura \\
    que para com teus filhos demonstraste: \\
    do céu nos deste um pão que é só doçura, \\
    e alimento do pobre te tornaste!
    \vspace{.1cm} \\
    \textcolor{red}{Ou:}
    \vspace{.1cm} \\
    Salve, ó corpo verdadeiro, \\
    que da Virgem Maria nasceste, \\
    e, salvando o mundo inteiro, \\
    sobre a cruz te ofereceste.
    \vspace{.1cm} \\
    Do teu lado, transpassado, \\
    sangue e água derramaste; \\
    sejas na morte provado \\
    por aqueles que salvaste!
    \vspace{.1cm} \\
    Jesus, fonte de alegria, \\
    alimento da unidade; \\
    Jesus, filho de Maria, \\
    Salvador da humanidade!
    \vspace{.1cm} \\
    \textcolor{red}{Ou:}
    \vspace{.1cm} \\
    Sou o pão que traz a vida, \\
    que por vós desceu dos céus: \\
    vive sempre quem se nutre \\
    deste pão, corpo de Deus.
    \vspace{.1cm} \\
    Dou ao mundo a minha carne, \\
    que da morte triunfou; \\
    dou aos homens o meu sangue, \\
    que aos escravos libertou.
    \vspace{.1cm} \\
    \textcolor{red}{O ministro, se possível acompanhado por uma pessoa que leva uma vela, aproxima-se dos doentes e diz uma só vez a todos que estejam no mesmo aposento ou a cada comungante:}
    \vspace{.1cm} \\
    Felizes os convidados para a Ceia do Senhor! \\
    Eis o Cordeiro de Deus, \\
    que tira o pecado do mundo.
    \vspace{.1cm} \\
    \textcolor{red}{Cada comungante acrescenta uma só vez:}
    \vspace{.1cm} \\
    Senhor, eu não sou digno(a) \\
    de que entres em minha morada, \\
    mas dizei uma palavra e serei salvo(a).
    \vspace{.1cm} \\
    \textcolor{red}{E recebe a Comunhão como de costume.}
    \vspace{.1cm} \\
    \textcolor{red}{O rito termina com a oração que pode ser recitada na igreja, na sacristia ou no último quarto.}
    \vspace{.1cm} \\
    Oremos.
    \vspace{.1cm} \\
    Senhor, Pai Santo, Deus todo-poderoso, \\
    nós vos pedimos confiantes \\
    que o sagrado Corpo (o sagrado Sangue) \\
    de vosso filho, nosso Senhor Jesus Cristo, \\
    seja para nosso irmão (nossa irmã) \\
    remédio de eternidade, \\
    tanto para o corpo como para a alma. \\
    Por Cristo, nosso Senhor.
    \vspace{.1cm} \\
    \textcolor{red}{Os presentes respondem:}
    \vspace{.1cm} \\
    Amém.
\end{flushleft}
\end{document}
