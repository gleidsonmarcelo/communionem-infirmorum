\documentclass{book}
\usepackage{import}
\import{/home/vatican/communionem-infirmorum/packages}{packages}
\geometry{a5paper, hdivide={1cm,*,1cm}, vdivide={1cm,*,1cm}}
\begin{document}
\pagestyle{empty}
\begin{center}
    \large Rito Ordinário da Comunhão dos Enfermos
\end{center}
\begin{center}
    \textbf{Ritos Iniciais}
\end{center}
\begin{flushleft}
    \textcolor{red}{O ministro, com veste conveniente a esta função, aproxima-se e saúda cordialmente o enfermo e todos os presentes, acrescentando, se for o caso, a seguinte saudação:}
    \vspace{.1cm} \\
    A paz esteja nesta casa e com todos os seus habitantes.
    \vspace{.1cm} \\
    \textcolor{red}{Podem-se usar também outras palavras da Sagrada Escritura, com as quais os fiéis costumam saudar-se. \\ Em seguida, depondo o Sacramento sobre a mesa, adora-o com todos os presente. \\ O ministro convida o doente e os demais presentes ao ato penitencial:}
    \vspace{.1cm} \\
    Irmãos e irmãs, \\
    reconheçamos os nossos pecados, \\
    para participarmos dignamente \\
    desta santa celebração.
    \vspace{.1cm} \\
    \textcolor{red}{Após um momento de silêncio, o ministro convida à confissão:}
    \vspace{.1cm} \\
    Confessemos os nossos pecados:
    \vspace{.1cm} \\
    \textcolor{red}{E todos prosseguem:}
    \vspace{.1cm} \\
    Confesso a Deus todo-poderoso \\
    e a vós, irmãos e irmãs, \\
    que pequei muitas vezes \\
    por pensamentos e palavras, \\
    atos e omissões,
    \vspace{.1cm} \\
    \textcolor{red}{E batendo no peito, dizem:}
    \vspace{.1cm} \\
    por minha culpa, minha tão grande culpa.
    \vspace{.1cm} \\
    \textcolor{red}{Em seguida, continuam:}
    \vspace{.1cm} \\
    E peço à Virgem Maria, aos anjos e santos e a \\
    vós, irmãos e irmãs, \\
    que roqueis por mim a Deus, nosso Senhor.
    \vspace{.1cm} \\
    \textcolor{red}{O ministro conclui:}
    \vspace{.1cm} \\
    Deus todo-poderoso tenha compaixão de nós, \\
    perdoe os nossos pecados \\
    e nos conduza à vida eterna.
    \vspace{.1cm} \\
    \textcolor{red}{todos respondem:}
    \vspace{.1cm} \\
    Amém.
    \vspace{.1cm} \\
    \textcolor{red}{Outras fórmulas de ato penitencial:}
    \vspace{.1cm} \\
    \textcolor{red}{O ministro convida os fiéis à penitencia:}
    \vspace{.1cm} \\
    Irmãos e irmãs, \\
    reconheçamos os nossos pecados, \\
    para participarmos dignamente \\
    desta santa celebração.
    \vspace{.1cm} \\
    \textcolor{red}{Após um momento de silêncio, o ministro diz:}
    \vspace{.1cm} \\
    Tende compaixão de nós, Senhor.
    \vspace{.1cm} \\
    \textcolor{red}{Todos respondem:}
    \vspace{.1cm} \\
    Porque somos pecadores.
    \vspace{.1cm} \\
    \textcolor{red}{O ministro:}
    \vspace{.1cm} \\
    Manifestai, Senhor, a vossa misericórdia.
    \vspace{.1cm} \\
    \textcolor{red}{Todos respondem:}
    \vspace{.1cm} \\
    E dai-nos a vossa salvação.
    \vspace{.1cm} \\
    \textcolor{red}{E o ministro conclui:}
    \vspace{.1cm} \\
    Deus todo-poderoso tenha compaixão de nós, \\
    perdoe os nossos pecados \\
    e nos conduza à vida eterna.
    \vspace{.1cm} \\
    \textcolor{red}{todos respondem:}
    \vspace{.1cm} \\
    Amém.
    \vspace{.1cm} \\
    \textcolor{red}{Ou:}
    \vspace{.1cm} \\
    \textcolor{red}{O ministro convida os fiéis à penitencia:}
    \vspace{.1cm} \\
    Irmãos e irmãs, \\
    reconheçamos os nossos pecados, \\
    para participarmos dignamente \\
    desta santa celebração.
    \vspace{.1cm} \\
    \textcolor{red}{Faz-se um momento de silêncio, Em seguida, o ministro ou algum dos presentes propõe as seguintes invocações ou outras semelhantes, como \textit{Senhor tente piedade de nós}:}
    \vspace{.1cm} \\
    Senhor, que pelo vosso mistério pascal \\
    nos obtivestes a salvação, \\
    tende piedade de nós.
    \vspace{.1cm} \\
    \textcolor{red}{Todos:}
    \vspace{.1cm} \\
    Senhor, tende piedade de nós.
    \vspace{.1cm} \\
    \textcolor{red}{Ministro:}
    \vspace{.1cm} \\
    Cristo, que não cessais de renovar entre nós \\
    as maravilhas da vossa paixão, \\
    tende piedade de nós.
    \vspace{.1cm} \\
    \textcolor{red}{Todos:}
    \vspace{.1cm} \\
    Cristo, tende piedade de nós.
    \vspace{.1cm} \\
    \textcolor{red}{Ministro:}
    \vspace{.1cm} \\
    Senhor, que pela recepção do vosso Corpo, \\
    nos tornais participantes do Sacrifício pascal, \\
    tende piedade de nós.
    \vspace{.1cm} \\
    \textcolor{red}{Todos:}
    \vspace{.1cm} \\
    Senhor, tende piedade de nós.
    \vspace{.1cm} \\
    \textcolor{red}{E o ministro conclui:}
    \vspace{.1cm} \\
    Deus todo-poderoso tenha compaixão de nós, \\
    perdoe os nossos pecados \\
    e nos conduza à vida eterna.
    \vspace{.1cm} \\
    \textcolor{red}{todos respondem:}
    \vspace{.1cm} \\
    Amém.
\end{flushleft}
\begin{center}
    \textbf{Breve Leitura da Palavra de Deus}
\end{center}
\begin{flushleft}
    \textcolor{red}{Se for conveniente, poderá ser lido por um dos presentes ou pelo próprio ministro um texto da Escritura, como, por exemplo:}
    \vspace{.1cm} \\
    \textcolor{red}{Jo 6,54--55}
    \vspace{.1cm} \\
    Quem come a minha carne \\
    e bebe meu sangue \\
    tem a vida eterna, \\
    e eu o ressuscitarei no último dia. \\
    Porque a minha carne é verdadeira comida \\
    e o meu sangue, verdadeira bebida.
    \vspace{.1cm} \\
    \textcolor{red}{Jo 6,54--58}
    \vspace{.1cm} \\
    Quem come a minha carne \\
    e bebe meu sangue \\
    tem a vida eterna, \\
    e eu o ressuscitarei no último dia. \\
    Porque a minha carne é verdadeira comida \\
    e o meu sangue, verdadeira bebida. \\
    Quem come a minha carne e bebe o meu sangue \\
    permanece em mim e eu nele. \\
    Como o Pai, que vive, me enviou, \\
    e eu vivo por causa do Pai, \\
    assim o que me come viverá por causa de mim. \\
    Este é o pão que desceu do céu. \\
    Não é como aquele que os vossos pais comeram. \\
    Eles morreram. \\
    Aquele que come este pão viverá para sempre.
    e o meu sangue, verdadeira bebida.
    \vspace{.1cm} \\
    \textcolor{red}{Jo 14,6}
    \vspace{.1cm} \\
    Eu sou o Caminho, a Verdade e a Vida. \\
    Ninguém vai ao Pai senão por mim.
    \vspace{.1cm} \\
    \textcolor{red}{Jo 14,23}
    \vspace{.1cm} \\
    Se alguém me ama, guardará a minha palavra, \\
    e o meu Pai o amará, \\
    e nós viremos e faremos nele a nossa morada.
    \vspace{.1cm} \\
    \textcolor{red}{Jo 15,4}
    \vspace{.1cm} \\
    Permanecei em mim \\
    e eu permanecerei em vós. \\
    Como o ramo não pode dar fruto por si mesmo, \\
    se não permanecer na videira, \\
    assim também vós não podereis dar fruto, \\
    se não permanecerdes em mim.
    \vspace{.1cm} \\
    \textcolor{red}{1Cor 11,26}
    \vspace{.1cm} \\
    Todas as vezes que comerdes deste pão \\
    e beberdes deste cálice, \\
    estareis proclamando a morte do Senhor, \\
    até que ele venha.
\end{flushleft}
\begin{center}
    \textbf{Sagrada Comunhão}
\end{center}
\begin{flushleft}
    \textcolor{red}{O ministro, com estas palavras ou outras semelhantes introduz a oração do Senho:}
    \vspace{.1cm} \\
    Agora, todos juntos, rezemos a Deus, como nosso Senhor Jesus Cristo nos ensinou:
    \vspace{.1cm} \\
    \textcolor{red}{E todos prosseguem juntos:}
    \vspace{.1cm} \\
    Pai nosso que estais nos céus, \\
    santificado seja o vosso nome; \\
    venha a nós o vosso reino, \\
    seja feita a vossa vontade, \\
    assim na terra como no céu; \\
    o pão nosso de cada dia nos dia hoje; \\
    perdoai-nos as nossas ofensas, \\
    assim como nós perdoamos \\
    a quem nos tem ofendido; \\
    e não nos deixeis cair em tentação, \\
    mas livrai-nos do mal.
    \vspace{.1cm} \\
    \textcolor{red}{O ministro apresenta o Santíssimo Sacramento, dizendo:}
    \vspace{.1cm} \\
    Felizes os convidados para a Ceia do Senhor!
    \vspace{.1cm} \\
    \textcolor{red}{Ou:}
    \vspace{.1cm} \\
    Provai e vede como o Senhor é bom; \\
    feliz de quem nele encontra seu refúgio. \\
    Eis o Cordeiro de Deus \\
    que tira o pecado ddo mundo.
    \vspace{.1cm} \\
    \textcolor{red}{O doente e os que forem comungar dizem um só vez:}
    \vspace{.1cm} \\
    Senhor, eu não sou digno(a) \\
    de que entreis em minha morada, \\
    mas dizei uma palavra e serei salvo(a).
    \vspace{.1cm} \\
    \textcolor{red}{O ministro aproxima-se do doente, apresenta-lhe o Sacramento e diz:}
    \vspace{.1cm} \\
    O Corpo de Cristo (ou: O Sangue de Cristo).
    \vspace{.1cm} \\
    \textcolor{red}{O doente responde:}
    \vspace{.1cm} \\
    Amém.
    \vspace{.1cm} \\
    \textcolor{red}{E recebe a Comunhão.}
    \vspace{.1cm} \\
    \textcolor{red}{As outras pessoas que vão comungar recebem a Comunhão como de costume.}
    \vspace{.1cm} \\
    \textcolor{red}{Depois da distribuição da Comunhão, o ministro faz purificação de costume. Se for conveniente, observe-se o silêncio sagrado por algum tempo.}
    \vspace{.1cm} \\
    \textcolor{red}{Em seguida, o ministro conclui com a oração:}
    \vspace{.1cm} \\
    Oremos.
    \vspace{.1cm} \\
    Senhor, Pai Santo, Deus todo-poderoso, \\
    nós vos pedimos confiantes \\
    que o sagrado Corpo (o sagrado Sangue) \\
    de vosso filho, nosso Senhor Jesus Cristo, \\
    seja para nosso irmão (nossa irmã) \\
    remédio de eternidade, \\
    tanto para o corpo como para a alma. \\
    Por Cristo, nosso Senhor.
    \vspace{.1cm} \\
    \textcolor{red}{Todos respondem:}
    \vspace{.1cm} \\
    Amém.
    \vspace{.1cm} \\
    \textcolor{red}{Outras orações à escolha:}
    \vspace{.1cm} \\
    Ó Deus, \\
    que pelo mistério pascal do vosso Filho \\
    Unigênito, \\
    levastes à plenitude a obra da salvação \\
    dos seres humanos, \\
    concedei-nos que, \\
    proclamando com fé a morte \\
    e a ressurreição do vosso Filho \\
    nos sinais do sacramento, \\
    sintamos crescer continuamente em nós \\
    a graça da vossa salvação. \\
    Por Cristo, nosso Senhor.
    \vspace{.1cm} \\
    \textcolor{red}{Ou:}
    \vspace{.1cm} \\
    Penetrai-nos, ó Deus, \\
    com o vosso Espírito de caridade, \\
    para que vivam unidos no vosso amor, \\
    os que alimentais com o mesmo pão. \\
    Por Cristo, nosso Senhor.
    \vspace{.1cm} \\
    \textcolor{red}{Ou:}
    \vspace{.1cm} \\
    Santificai-nos, ó Deus, \\
    pela comunhão à vossa mesa, \\
    para que o Corpo, e o Sangue de Cristo \\
    unam todos os irmãos e irmãs.
    Por Cristo, nosso Senhor.
    \vspace{.1cm} \\
    \textcolor{red}{Ou:}
    \vspace{.1cm} \\
    Alimentados pelo pão espiritual, \\
    nós vos suplicamos, ó Deus, \\
    que pela participação nesta Eucaristia, \\
    nos ensinais a julgar com sabedoria \\
    os valores terrenos, \\
    e colocar nossas esperanças nos bens eternos. \\
    Por Cristo, nosso Senhor.
    \vspace{.1cm} \\
    \textcolor{red}{Ou:}
    \vspace{.1cm} \\
    Nós comungamos, Senhor Deus, \\
    no mistério da vossa glória, \\
    e nos empenhamos em render-vos graças, \\
    porque nos concedeis, ainda na terra, \\
    participar das coisas do céu. \\
    Por Cristo, nosso Senhor.
    \vspace{.1cm} \\
    \textcolor{red}{Ou:}
    \vspace{.1cm} \\
    Deus todo-poderoso, \\
    que refazeis as nossas forças \\
    pelos vossos sacramentos, \\
    nós vos suplicamos a graça de vos servir \\
    por uma vida que vos agrade. \\
    Por Cristo, nosso Senhor.
    \vspace{.1cm} \\
    \textcolor{red}{Ou:}
    \vspace{.1cm} \\
    Ó Deus, vós quisestes que participássemos \\
    do mesmo Pão e do mesmo Cálice; \\
    fazei-nos viver de tal modo unidos em Cristo, \\
    que tenhamos a alegria de produzir muitos frutos \\
    para a salvação do mundo. \\
    Por Cristo, nosso Senhor.
    \vspace{.1cm} \\
    \textcolor{red}{Ou:}
    \vspace{.1cm} \\
    Restaurados à vossa mesa pelo Pão da vida, \\
    nós vos pedimos, ó Deus, \\
    que este alimento da caridade \\
    fortifique os nossos corações \\
    e nos leve a vos servir em nossos irmão e irmãs. \\
    Por Cristo, nosso Senhor.
    \vspace{.1cm} \\
    \textcolor{red}{Ou:}
    \vspace{.1cm} \\
    Fortificados por este alimento sagrado, \\
    nós vos damos graças, ó Deus, \\
    e imploramos vossa clemência; \\
    fazei que perseverem na sinceridade \\
    do vosso amor \\
    aqueles que fortalecestes pela infusão \\
    do Espírito Santo. \\
    Por Cristo, nosso Senhor.
    \vspace{.1cm} \\
    \textcolor{red}{Ou:}
    \vspace{.1cm} \\
    Alimentados com o mesmo Pão, \\
    nós vos pedimos, ó Deus, \\
    que possamos viver uma vida nova \\
    e perseverar no vosso amor. \\
    Por Cristo, nosso Senhor.
    \vspace{.1cm} \\
    \textcolor{red}{No tempo pascal, dize-se uma das seguintes orações:}
    \vspace{.1cm} \\
    Ó Deus, derramai em nós \\
    o vosso Espírito de caridade, \\
    para que, saciados pelos sacramentos pascais, \\
    permaneçamos unidos no vosso amor. \\
    Por Cristo, nosso Senhor.
    \vspace{.1cm} \\
    \textcolor{red}{Ou:}
    \vspace{.1cm} \\
    Purificados da antiga culpa, \\
    nós vos pedimos, ó Deus, \\
    que a comunhão no Sacramento do vosso Filho \\
    nos transforme em nova criatura.
    Por Cristo, nosso Senhor.
    \vspace{.1cm} \\
    \textcolor{red}{Ou:}
    \vspace{.1cm} \\
    Deus eterno e todo-poderoso, \\
    que pela ressurreição de Cristo \\
    nos renovais para a vida eterna, \\
    fazei frutificar em nós o Sacramento pascal, \\
    e infundi em nossos corações \\
    a fortaleza deste sacramento salutar. \\
    Por Cristo, nosso Senhor.
\end{flushleft}
\begin{center}
    \textbf{Ritos Finais}
\end{center}
\begin{flushleft}
    \textcolor{red}{O ministro, invocando a bênçao de Deus, persigna-se, dizendo:}
    \vspace{.1cm} \\
    Que o Senhor nos abençoe, \\
    guarde-nos de todo o mal \\
    e nos conduza à vida eterna.
    \vspace{.1cm} \\
    \textcolor{red}{Ou:}
    \vspace{.1cm} \\
    O Senhor todo-poderoso e cheio de misericórdia, \\
    Pai e Filho e Espírito Santo, \\
    nos abençoe e nos guarde.
    \vspace{.1cm} \\
    \textcolor{red}{Todos respondem:}
    \vspace{.1cm} \\
    Amém.
\end{flushleft}
\end{document}
