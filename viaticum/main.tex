\documentclass{book}
\usepackage{import}
\import{/home/vatican/communionem-infirmorum/packages}{packages}
\geometry{a5paper, hdivide={1cm,*,1cm}, vdivide={1cm,*,1cm}}
\begin{document}
\pagestyle{empty}
\begin{center}
    \large Viático
\end{center}
\begin{center}
    \textbf{Ritos Iniciais}
\end{center}
\begin{flushleft}
    \textcolor{red}{O ministro, revestido com veste conveniente a esta função, aproxima-se e saúda cordialmente o enfermo e todos os presentes, acrescentando, se for o caso, a seguinte saudação:}
    \vspace{.1cm} \\
    A paz esteja nesta casa e com todos os seus habitantes.
    \vspace{.1cm} \\
    \textcolor{red}{Podem-se usar também outras palavras da Sagrada Escritura, com as quais os fiéis costumam saudar-se. \\ Em seguida, depondo o Sacramento sobre a mesa, adora-o com todos os presentes.}
    \vspace{.1cm} \\
    \textcolor{red}{Dirige, então, aos presentes esta exortação ou outra mais adaptada às condições do doente:}
    \vspace{.1cm} \\
    Caros irmãos e irmãs: \\
    Nosso Senhor Jesus Cristo, \\
    antes de passar deste mundo para o Pai, \\
    deixou-nos o sacramento do seu Corpo \\
    e do seu Sangue, \\
    para que, na hora da nossa passagem \\
    desta vida para ele, \\
    fortificados por esse alimento da último viagem, \\
    nos encontrássemos munidos com o penhor \\
    da ressurreição. \\
    Unidos pela caridade ao nosso irmão \\
    (à nossa irmã), rezemos por ele(a).
    \vspace{.1cm} \\
    \textcolor{red}{E todos rezam por algum tempo em silêncio.}
    \vspace{.1cm} \\
    \textcolor{red}{O ministro convida o enfermo e os demais presentes ao ato penitencial:}
    \vspace{.1cm} \\
    Irmãos e irmãs, \\
    reconheçamos os nossos pecados, \\
    para participarmos dignamente \\
    desta santa celebração.
    \vspace{.1cm} \\
    \textcolor{red}{Após um momento de silêncio, o ministro convida à confissão:}
    \vspace{.1cm} \\
    Confessemos os nossos pecados:
    \vspace{.1cm} \\
    \textcolor{red}{E todos prosseguem:}
    \vspace{.1cm} \\
    Confesso a Deus todo-poderoso \\
    e a vós, irmãos e irmãs, \\
    que pequei muitas vezes \\
    por pensamentos e palavras, \\
    atos e omissões,
    \vspace{.1cm} \\
    \textcolor{red}{E batendo no peito, dizem:}
    \vspace{.1cm} \\
    por minha culpa, minha tão grande culpa.
    \vspace{.1cm} \\
    \textcolor{red}{Em seguida, continuam:}
    \vspace{.1cm} \\
    E peço à Virgem Maria, aos anjos e santos e a \\
    vós, irmãos e irmãs, \\
    que roqueis por mim a Deus, nosso Senhor.
    \vspace{.1cm} \\
    \textcolor{red}{O ministro conclui:}
    \vspace{.1cm} \\
    Deus todo-poderoso tenha compaixão de nós, \\
    perdoe os nossos pecados \\
    e nos conduza à vida eterna.
    \vspace{.1cm} \\
    \textcolor{red}{todos respondem:}
    \vspace{.1cm} \\
    Amém.
    \vspace{.1cm} \\
    \textcolor{red}{Outras fórmulas de ato penitencial:}
    \vspace{.1cm} \\
    \textcolor{red}{O ministro convida os fiéis à penitencia:}
    \vspace{.1cm} \\
    Irmãos e irmãs, \\
    reconheçamos os nossos pecados, \\
    para participarmos dignamente \\
    desta santa celebração.
    \vspace{.1cm} \\
    \textcolor{red}{Após um momento de silêncio, o ministro diz:}
    \vspace{.1cm} \\
    Tende compaixão de nós, Senhor.
    \vspace{.1cm} \\
    \textcolor{red}{Todos respondem:}
    \vspace{.1cm} \\
    Porque somos pecadores.
    \vspace{.1cm} \\
    \textcolor{red}{O ministro:}
    \vspace{.1cm} \\
    Manifestai, Senhor, a vossa misericórdia.
    \vspace{.1cm} \\
    \textcolor{red}{Todos respondem:}
    \vspace{.1cm} \\
    E dai-nos a vossa salvação.
    \vspace{.1cm} \\
    \textcolor{red}{E o ministro conclui:}
    \vspace{.1cm} \\
    Deus todo-poderoso tenha compaixão de nós, \\
    perdoe os nossos pecados \\
    e nos conduza à vida eterna.
    \vspace{.1cm} \\
    \textcolor{red}{todos respondem:}
    \vspace{.1cm} \\
    Amém.
    \vspace{.1cm} \\
    \textcolor{red}{Ou:}
    \vspace{.1cm} \\
    \textcolor{red}{O ministro convida os fiéis à penitencia:}
    \vspace{.1cm} \\
    Irmãos e irmãs, \\
    reconheçamos os nossos pecados, \\
    para participarmos dignamente \\
    desta santa celebração.
    \vspace{.1cm} \\
    \textcolor{red}{Faz-se um momento de silêncio, Em seguida, o ministro ou algum dos presentes propõe as seguintes invocações ou outras semelhantes, como \textit{Senhor tente piedade de nós}:}
    \vspace{.1cm} \\
    Senhor, que pelo vosso mistério pascal \\
    nos obtivestes a salvação, \\
    tende piedade de nós.
    \vspace{.1cm} \\
    \textcolor{red}{Todos:}
    \vspace{.1cm} \\
    Senhor, tende piedade de nós.
    \vspace{.1cm} \\
    \textcolor{red}{Ministro:}
    \vspace{.1cm} \\
    Cristo, que não cessais de renovar entre nós \\
    as maravilhas da vossa paixão, \\
    tende piedade de nós.
    \vspace{.1cm} \\
    \textcolor{red}{Todos:}
    \vspace{.1cm} \\
    Cristo, tende piedade de nós.
    \vspace{.1cm} \\
    \textcolor{red}{Ministro:}
    \vspace{.1cm} \\
    Senhor, que pela recepção do vosso Corpo, \\
    nos tornais participantes do Sacrifício pascal, \\
    tende piedade de nós.
    \vspace{.1cm} \\
    \textcolor{red}{Todos:}
    \vspace{.1cm} \\
    Senhor, tende piedade de nós.
    \vspace{.1cm} \\
    \textcolor{red}{E o ministro conclui:}
    \vspace{.1cm} \\
    Deus todo-poderoso tenha compaixão de nós, \\
    perdoe os nossos pecados \\
    e nos conduza à vida eterna.
    \vspace{.1cm} \\
    \textcolor{red}{todos respondem:}
    \vspace{.1cm} \\
    Amém.
\end{flushleft}
\begin{center}
    \textbf{Breve Leitura da Palavra de Deus}
\end{center}
\begin{flushleft}
    \textcolor{red}{Se for conveniente, poderá ser lido por um dos presentes ou pelo próprio ministro um texto da Escritura, como, por exemplo:}
    \vspace{.1cm} \\
    \textcolor{red}{Jo 6,54--55}
    \vspace{.1cm} \\
    Quem come a minha carne \\
    e bebe meu sangue \\
    tem a vida eterna, \\
    e eu o ressuscitarei no último dia. \\
    Porque a minha carne é verdadeira comida \\
    e o meu sangue, verdadeira bebida.
    \vspace{.1cm} \\
    \textcolor{red}{Jo 6,54--58}
    \vspace{.1cm} \\
    Quem come a minha carne \\
    e bebe meu sangue \\
    tem a vida eterna, \\
    e eu o ressuscitarei no último dia. \\
    Porque a minha carne é verdadeira comida \\
    e o meu sangue, verdadeira bebida. \\
    Quem come a minha carne e bebe o meu sangue \\
    permanece em mim e eu nele. \\
    Como o Pai, que vive, me enviou, \\
    e eu vivo por causa do Pai, \\
    assim o que me come viverá por causa de mim. \\
    Este é o pão que desceu do céu. \\
    Não é como aquele que os vossos pais comeram. \\
    Eles morreram. \\
    Aquele que come este pão viverá para sempre.
    e o meu sangue, verdadeira bebida.
    \vspace{.1cm} \\
    \textcolor{red}{Jo 14,6}
    \vspace{.1cm} \\
    Eu sou o Caminho, a Verdade e a Vida. \\
    Ninguém vai ao Pai senão por mim.
    \vspace{.1cm} \\
    \textcolor{red}{Jo 14,23}
    \vspace{.1cm} \\
    Se alguém me ama, guardará a minha palavra, \\
    e o meu Pai o amará, \\
    e nós viremos e faremos nele a nossa morada.
    \vspace{.1cm} \\
    \textcolor{red}{Jo 14,27}
    \vspace{.1cm} \\
    Deixo-vos a paz, \\
    a minha paz vos dou; \\
    mas não a dou como o mundo. \\
    Não se perturbe nem se intimide o vosso coração.
    \vspace{.1cm} \\
    \textcolor{red}{Jo 15,4}
    \vspace{.1cm} \\
    Permanecei em mim \\
    e eu permanecerei em vós. \\
    Como o ramo não pode dar fruto por si mesmo, \\
    se não permanecer na videira, \\
    assim também vós não podereis dar fruto, \\
    se não permanecerdes em mim.
    \vspace{.1cm} \\
    \textcolor{red}{Jo 15,5}
    \vspace{.1cm} \\
    Eu sou a videira e vós os ramos. \\
    Aquele que permaneceu em mim, e eu nele, \\
    esse produz muito fruto; \\
    porque sem mim nada podeis fazer.
    \vspace{.1cm} \\
    \textcolor{red}{1Cor 11,26}
    \vspace{.1cm} \\
    Todas as vezes que comerdes deste pão \\
    e beberdes deste cálice, \\
    estareis proclamando a morte do Senhor, \\
    até que ele venha.
    \vspace{.1cm} \\
    \textcolor{red}{1Jo 4,16}
    \vspace{.1cm} \\
    Também nós conhecemos o amor \\
    que Deus tem para conosco, \\
    e acreditamos nele. \\
    Deus é amor: \\
    quem permanece no amor, \\
    permanece com Deus, \\
    e Deus permaneceu com ele.
\end{flushleft}
\begin{center}
    \textbf{Profissão de fé batismal}
\end{center}
\begin{flushleft}
    \textcolor{red}{Convém que o enfermo, antes de receber o Viático, renove a profissão de é batismal. Portanto, o ministro, após breve introdução com palavras adequadas, interroga:}
    \vspace{.1cm} \\
    Crês em Deus Pai todo-poderoso, criador do céu e da terra?
    \vspace{.1cm} \\
    {\color{red} \Rbar.} Creio.
    \vspace{.1cm} \\
    Crês em Jesus Cristo, seu único Filho, \\
    nosso Senhor, \\
    que nasceu da Virgem Maria, \\
    padeceu e foi sepultado, \\
    ressuscitou dos mortos e subiu ao céu?
    \vspace{.1cm} \\
    {\color{red} \Rbar.} Creio.
    \vspace{.1cm} \\
    Crês no Espírito Santo, \\
    na santa Igreja católica, \\
    na comunhão dos santos, \\
    na remissão dos pecados, \\
    na ressurreição dos mortos \\
    e na vida eterna?
    \vspace{.1cm} \\
    {\color{red} \Rbar.} Creio.
    \vspace{.1cm} \\
\end{flushleft}
\begin{center}
    \textbf{Preces pelo enfermo}
\end{center}
\begin{flushleft}
    \textcolor{red}{Em seguida, se as condições do enfermo o permitirem, faz-se uma breve súplica, com estas palavras ou outras semelhantes, a que o doente responderá, quanto possível, com todos os presentes:}
    \vspace{.1cm} \\
    Caros irmãos e irmãs, \\
    invoquemos num só coração \\
    nosso Senhor Jesus Cristo:
    \vspace{.1cm} \\
    --- Senhor, que nos amastes até o fim, \\
    e vos entregastes à morte para nos dar a vida, \\
    nós vos rogamos por nosso(a) irmão(ã) \textcolor{red}{N.}
    \vspace{.1cm} \\
    {\color{red} \Rbar.} Senhor, escutai a nossa prece.
    \vspace{.1cm} \\
    --- Senhor, que dissestes: \\
    Quem come a minha Carne \\
    e bebe o meu Sangue possui a vida eterna, \\
    nós vos rogamos por nosso(a) irmão(ã) \textcolor{red}{N.}
    \vspace{.1cm} \\
    {\color{red} \Rbar.} Senhor, escutai a nossa prece.
    \vspace{.1cm} \\
    --- Senhor, que nos convidais ao banquete \\
    onde não haverá mais dor nem pranto \\
    nem tristeza nem separação, \\
    nós vos rogamos por nosso(a) irmão(ã) \textcolor{red}{N.}
    \vspace{.1cm} \\
    {\color{red} \Rbar.} Senhor, escutai a nossa prece.
    \vspace{.1cm} \\
\end{flushleft}
\begin{center}
    \textbf{Viático}
\end{center}
\begin{flushleft}
    \textcolor{red}{O ministro, com estas palavras ou outras semelhantes introduz a oração do Senhor:}
    \vspace{.1cm} \\
    Agora, todos juntos, rezemos a Deus, como nosso Senhor Jesus Cristo nos ensinou:
    \vspace{.1cm} \\
    \textcolor{red}{E todos prosseguem juntos:}
    \vspace{.1cm} \\
    Pai nosso que estais nos céus, \\
    santificado seja o vosso nome; \\
    venha a nós o vosso reino, \\
    seja feita a vossa vontade, \\
    assim na terra como no céu; \\
    o pão nosso de cada dia nos dia hoje; \\
    perdoai-nos as nossas ofensas, \\
    assim como nós perdoamos \\
    a quem nos tem ofendido; \\
    e não nos deixeis cair em tentação, \\
    mas livrai-nos do mal.
    \vspace{.1cm} \\
    \textcolor{red}{O ministro apresenta o Santíssimo Sacramento, dizendo:}
    \vspace{.1cm} \\
    Felizes os convidados para a Ceia do Senhor! \\
    Eis o Cordeiro de Deus \\
    que tira o pecado ddo mundo.
    \vspace{.1cm} \\
    \textcolor{red}{O doente, se puder, e os outros que forem comungar dizem:}
    \vspace{.1cm} \\
    Senhor, eu não sou digno(a) \\
    de que entreis em minha morada, \\
    mas dizei uma palavra e serei salvo(a).
    \vspace{.1cm} \\
    \textcolor{red}{O ministro aproxima-se do doente, apresenta-lhe o Sacramento e diz:}
    \vspace{.1cm} \\
    O Corpo de Cristo (ou: O Sangue de Cristo).
    \vspace{.1cm} \\
    \textcolor{red}{O doente responde:}
    \vspace{.1cm} \\
    Amém.
    \vspace{.1cm} \\
    \textcolor{red}{E, imediatamente ou depois de ter dado a Comunhão, o ministro acrescenta:}
    \vspace{.1cm} \\
    Que ele te guarde e te conduza à vida eterna!
    \vspace{.1cm} \\
    \textcolor{red}{O doente responde:}
    \vspace{.1cm} \\
    Amém.
    \vspace{.1cm} \\
    \textcolor{red}{Aos presentes que desejam comungar será dada a Comunhão como de costume:}
    \vspace{.1cm} \\
    \textcolor{red}{Terminada a distribuição da Comunhão, o ministro faz a purificação de costume. Se for conveniente, observe-se por algum tempo o silêncio sagrado.}
\end{flushleft}
\begin{center}
    \textbf{Ritos Finais}
\end{center}
\begin{flushleft}
    \textcolor{red}{A seguir, o ministro conclui com a oração:}
    \vspace{.1cm} \\
    Oremos.
    \vspace{.1cm} \\
    Ó Deus, em vosso Filho temos o caminho, \\
    a verdade e a vida; olhai com bondade \\
    o(a) vosso(a) servo(a) \textcolor{red}{N.} e fazei que, \\
    confiando em vossas promessas e renovados(a) \\
    pelo Corpo e o Sangue do vosso Filho, \\
    caminhe em paz para o vosso reino. \\
    Por Cristo, nosso Senhor.
    \vspace{.1cm} \\
    \textcolor{red}{Todos respondem:}
    \vspace{.1cm} \\
    Amém.
    \vspace{.1cm} \\
    \textcolor{red}{Outra oração à escolha:}
    \vspace{.1cm} \\
    Ó Deus, salvação dos que creem em vós, \\
    concedei que o(a) vosso(a) filho(a) \textcolor{red}{N.}, \\
    confortado(a) pelo Pão e o Vinho celestes, \\
    possa chegar ao reino da luz e da vida. \\
    Por Cristo, nosso Senhor.
    \vspace{.1cm} \\
    \textcolor{red}{Em seguida, o ministro diz:}
    \vspace{.1cm} \\
    Que Deus esteja sempre contigo, \\
    te proteja com seu poder e te guarde em paz.
    \vspace{.1cm} \\
    \textcolor{red}{Por fim, o ministro e os demais presentes podem saudar o enfermo desejando-lhe a paz.}
\end{flushleft}
\end{document}
